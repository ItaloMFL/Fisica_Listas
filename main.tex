\documentclass[a4paper, 12pt]{article}

\usepackage{graphicx} % Required for inserting images
\usepackage[utf8]{inputenc}
\usepackage[brazil]{babel}
\usepackage[outline]{contour} % glow around text
\usepackage{lipsum}
\usepackage{amsmath} % For align environment

\title{Segunda Lista de Fisica I}
\author{Italo Leite}
\date{Agosto de 2024}

\begin{document}
	
	\maketitle
	
	\begin{flushleft}
		\textbf{1. } Normalmente é possível fazer uma viagem de carro de San Diego a Los Angeles com velocidade média de 105 km/h, em 2h20min. Em uma tarde de sexta-feira, contudo, o trânsito está muito pesado e você percorre a mesma distância com um velocidade média de 70 km/h. calcule o tempo que você leva nesse percurso.
		
		\textbf{reposta:}
		
		Primeiro precisamos identificar a distância que ele percorre ao viajar por 2h20min em um velocidade média de 105 km/h. Para isso vamos utilizar a equação horária do espaço.
		
		\begin{equation*}
			s(t) = s_0 + v.t
		\end{equation*}
		
		Pegando de referência que o que o espaço inicial seja zero, então podemos fazer com que $s_0 = 0$ e ignorarmos ele na questão, deixando a equação da seguinte forma:
		
		\begin{equation*}
			s(t) = v.t
		\end{equation*}
		
		Onde o v é a velocidade média do automóvel, que é 105 km/h e o t é o tempo, que é 3h20min. Mas não podemos fazer essa multiplicação, pois as horas ou devem estar apenas em horas ou apenas em minutos.
		
		A hora dada pela questão é 2h20min. Como dito anteriormente, precisamos deixar apenas em horas ou apenas em minutos. Como a velocidade é km/h, a melhor opção é deixa-la em horas. Então
		
		\begin{equation*}
			t_1=2h \times \frac{60 min}{1 h} 
		\end{equation*}
		\begin{equation*}
			t_1 = 120 min 
		\end{equation*}
		
		Onde, $t_1$ é o tempo de 2horas para minutos da viagem.
		
		Como achamos os minutos de 2 horas, podemos conseguir os minutos totais apenas somando com os 20 minutos restantes, achando assim o tempo total.
		
		\begin{equation*}
			t_t = t_1 + t_2 \Rightarrow t_t = 120 min + 20 min
		\end{equation*}
		\begin{equation*}
			t_t = 140 min
		\end{equation*}
		
		
		Onde, $t_t$ é o tempo total da viagem, $t_1$ é o tempo achado logo acima e $t_2$ são os 20min restantes que sobranram das 2h20min. 
		
		Podemos transformar agora esses 140 min em horas para fazer a equação seguinte
		
		\begin{equation*}
			t_{th}= 140 min \times \frac{1h}{60 min} \Rightarrow 2,33 h 
		\end{equation*}
		
		Onde $t_{th}$ é o tempo total em horas. Substituindo as horas e a velocidade média na equação horária do espaço fica da seguinte forma:
		
		\begin{equation*}
			s(2,33h) =105 km/h \times 2,33h
		\end{equation*}
		\begin{equation*}
			s(2,33h) =105 \times 2,33 \times \frac{km}{h} \times \frac{h}{1}
		\end{equation*}
		\begin{equation*}
			s(2,33h) \approx 244,65 km
		\end{equation*}
		
		Sabendo a distância total da viagem, podemos substis o s(t) por 244,65 km e a velocidade do automóvel por 70 km/h. Deixando a equação da seguinte forma:
		
		\begin{equation*}
			244,65 km = 70 km/h\times t 
		\end{equation*}
		
		Isolando o t para achar o tempo.
		
		\begin{equation*}
			t = \frac{244,65 km}{70 km/h} 
		\end{equation*}
		\begin{equation*}
			t = \frac{244,65 km}{70 km/h} \times \frac{km}{1} \times \frac{h}{km} 
		\end{equation*}
		\begin{equation*}
			t = 3,5h 
		\end{equation*}
		O tempo total de viajem seria de 3,5 horas se ele viajar a uma velocidade média de 70 km/h. Ou seja, seria de 3h30min de viagem.
		
	\end{flushleft}
	
	
\end{document}
