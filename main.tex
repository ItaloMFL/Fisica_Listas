\documentclass[a4paper, 12pt]{article}

\usepackage{graphicx} % Required for inserting images
\usepackage[utf8]{inputenc}
\usepackage[brazil]{babel}
\usepackage[outline]{contour} % glow around text
\usepackage{lipsum}
\usepackage{amsmath} % For align environment
\usepackage{cancel}
\usepackage{blindtext}
\usepackage{tikz, tkz-base, tkz-fct}
\usepackage{pgfplots}

\usepackage{physics}
\usetikzlibrary{angles,quotes} % for pic
\contourlength{1.2pt}

\tikzset{>=latex} % for LaTeX arrow head
\usepackage{xcolor}
\colorlet{veccol}{green!70!black}
\colorlet{vcol}{green!70!black}
\colorlet{xcol}{blue!85!black}
\colorlet{projcol}{xcol!60}
\colorlet{unitcol}{xcol!60!black!85}
\colorlet{myblue}{blue!70!black}
\colorlet{myred}{red!90!black}
\colorlet{mypurple}{blue!50!red!80!black!80}
\tikzstyle{vector}=[->,very thick,xcol]

% Estou colocando 2 de espaço por cada pergunta

% Personalizando as coisas do latex
\usepackage[left=2.5cm,top=2.5cm,right=2.5cm,bottom=2.5cm]{geometry}

\title{Segunda Lista de Fisica I}
\author{Italo Leite}
\date{Agosto de 2024}

\begin{document}
	
	\maketitle
	
	\begin{flushleft}
		\textbf{1. } Normalmente é possível fazer uma viagem de carro de San Diego a Los Angeles com velocidade média de 105 km/h, em 2h20min. Em uma tarde de sexta-feira, contudo, o trânsito está muito pesado e você percorre a mesma distância com um velocidade média de 70 km/h. calcule o tempo que você leva nesse percurso.
		
		\textbf{reposta:}
		
		Primeiro precisamos identificar a distância que ele percorre ao viajar por 2h20min em um velocidade média de 105 km/h. Para isso vamos utilizar a equação horária do espaço.
		
		\begin{equation*}
			s(t) = s_0 + v.t
		\end{equation*}
		
		Pegando de referência que o que o espaço inicial seja zero, então podemos fazer com que $s_0 = 0$ e ignorarmos ele na questão, deixando a equação da seguinte forma:
		
		\begin{equation*}
			s(t) = v.t
		\end{equation*}
		
		Onde o v é a velocidade média do automóvel, que é 105 km/h e o t é o tempo, que é 3h20min. Mas não podemos fazer essa multiplicação, pois as horas ou devem estar apenas em horas ou apenas em minutos.
		
		A hora dada pela questão é 2h20min. Como dito anteriormente, precisamos deixar apenas em horas ou apenas em minutos. Como a velocidade é km/h, a melhor opção é deixa-la em horas. Então
		
		\begin{equation*}
			t_1=2 \cancel{h} \times \frac{60 min}{1 \cancel{h}} 
		\end{equation*}
		\begin{equation*}
			t_1 = 120 min 
		\end{equation*}
		
		Onde, $t_1$ é o tempo de 2horas para minutos da viagem.
		
		Como achamos os minutos de 2 horas, podemos conseguir os minutos totais apenas somando com os 20 minutos restantes, achando assim o tempo total.
		
		\begin{equation*}
			t_t = t_1 + t_2 \Rightarrow t_t = 120 min + 20 min
		\end{equation*}
		\begin{equation*}
			t_t = 140 min
		\end{equation*}
		
		
		Onde, $t_t$ é o tempo total da viagem, $t_1$ é o tempo achado logo acima e $t_2$ são os 20min restantes que sobranram das 2h20min. 
		
		Podemos transformar agora esses 140 min em horas para fazer a equação seguinte
		
		\begin{equation*}
			t_{th}= 140 \cancel{min} \cdot \frac{1h}{60 \cancel{min}} \Rightarrow 2,33 h 
		\end{equation*}
		
		Onde $t_{th}$ é o tempo total em horas. Substituindo as horas e a velocidade média na equação horária do espaço fica da seguinte forma:
		
		\begin{equation*}
			s(2,33h) =105 km/h \cdot 2,33h
		\end{equation*}
		\begin{equation*}
			s(2,33h) =105 \cdot 2,33 \cdot \frac{km}{\cancel{h}} \cdot \frac{\cancel{h}}{1}
		\end{equation*}
		\begin{equation*}
			s(2,33h) \approx 244,65 km
		\end{equation*}
		
		Sabendo a distância total da viagem, podemos substis o s(t) por 244,65 km e a velocidade do automóvel por 70 km/h. Deixando a equação da seguinte forma:
		
		\begin{equation*}
			244,65 km = 70 km/h \cdot t 
		\end{equation*}
		
		Isolando o t para achar o tempo.
		
		\begin{equation*}
			t = \frac{244,65 km}{70 km/h} 
		\end{equation*}
		\begin{equation*}
			t = \frac{244,65}{70} \cdot \frac{\cancel{km}}{1} \cdot \frac{h}{\cancel{km}} 
		\end{equation*}
		\begin{equation*}
			t = 3,5h 
		\end{equation*}
		O tempo total de viajem seria de 3,5 horas. Ou seja, seria de 3h30min de viagem.
		
		\vspace{2em}
		
		\textbf{2.} Dois corredores partem simultaneamente do mesmo ponto de uma pista circular de 200 m e correm em direções opostas. Um corre com uma velocidade constante de 6,20 m/s e o outro corre com uma velocidade  constante de 5,50 m/s. Quando eles se cruzam pela primeira vez, calcule: a) Por quanto tempo estão correndo; e, b) qual é a distância percorrida por cada um deles.
		
		\textbf{respsota: }
		
		A) Para resolver, precisamos imaginar uma linha retilínea que tem uma distância de 200 m. O corredor 1 vai começar na distância 0m e o segundo corredor irá começar na distância é de 200m. Como há apenas um movimento retilíneo uniforme, vamos utilizar a equação horária do espaço, que descrevemos da seguinte forma:
		
		\begin{equation*}
			s(t) = s_0 + v \cdot t
		\end{equation*}
		
		Onde o $s_0$ é o espaço inicial, $v$ é a velocidade constante e t é o dominio da função da imagem s(t).
		
		Pelo ponto de referência do x é crescente para a direita, o corredor dois irá andar negativamente em relação a trajetória.
		
		Fórmula para o corredor 1:
		
		\begin{equation*}
			s_{c1}(t) = 6,20m/s \cdot t
		\end{equation*}
		
		Fórmula para o corredor 2
		\begin{equation*}
			s_{c2}(t) = 200m - 5,50m/s \cdot t
		\end{equation*}
		
		Para descorbri por quanto tempo eles correram para ambos se cruzarem, preciamos igualar as duas equações $s_{c1} = s_{c2}$, a equação deve ficar da seguinte forma:
		
		\begin{equation*}
			6,20m/s \cdot t = 200m - 5,50m/s \cdot t
		\end{equation*}
		
		Precisamos isolar o t e resolver a divisão para descobrir o tempo que permaneceram correndo.
		
		\begin{equation*}
			t= \frac{200m}{11,70m/s} \approx 17,09s
		\end{equation*}
		
		O tempo de corrida foi aproximadamente 17,09 segundos.
		
		B) Com o tempo de corrida de ambos, vou considerar que ambos iram sair do espaço 0 (zero) e vão correr ou caminhar positivamente.
		
		\begin{equation*}
			s_{c1}(17,09s) = 6,20m/s \cdot 17,09s \approx 105,96m
		\end{equation*}
		\begin{equation*}
			s_{c2}(17,09s) = 5,50m/s \cdot 17,09s \approx 93,99m
		\end{equation*}
		
		Para saber se os calculos estão corretos, basta somar ambas distâncias se pe igual a 200m ou próximo de 200m.
		
		\begin{equation*}
			s_{t} = s_{c1} + s_{c2} = 105,96m + 93,99m = 199,955m
		\end{equation*}
		
		O erro acontece pois houve arredondamentos nas casas de decimais. Mas se tivesse feito os calculos com a precissão certa de cada número, o resultado seria 200m corretamente.
		
		
		\vspace{2em}
		
		
		\textbf{3.} Um carro para em um semáforo. A seguir ele percorre um trecho retilíneo de modo que sua distância ao sinal é dada por $x(t) = bt^2 - ct^3$, onde $b = 2,40m/s^2$ e $c = 0,120m/s^3$. A) Calcule a velocidade média do carro para o intervalo de tempo t = 0 até t = 10,0 s. b) calcule a velocidade instantânea do carro para i) t = 0; ii) t= 5,0 s; iii) t = 10,0 s. c) quanto tempo após o repouso o carro retorna novamente ao repouso?
		
		\textbf{resposta:}
		
		A) Para calcular a velocidade média precisa da seguinte equação
		
		\begin{equation*}
			v_m = \frac{\Delta{x}}{\Delta{t}} = \frac{x_f - x_i}{t_f - t_i}
		\end{equation*}
		
		Onde $x_f$ é a posição final do corpo, $x_i$ é a posição inicial do corpo, $t_f$ é o tempo final e $t_i$ é o tempo inicial. Agora precisamo calcular a velocidade média entre entre os intervalos t=0s até t=10s. Como não há uma termo idependente na questão, podemos ignorar o t=0s, pois vai dar algo menos 0 (zero).
		
		\begin{equation*}
			v_m = \frac{x(10s) - x(0s)}{10s - 0s} \Rightarrow \frac{x(10s)}{10s}
		\end{equation*}
		\begin{equation*}
			v_m = \frac{2,40m/s^2 \cdot (10s)^2 - 0,120m/s^3 \cdot (10s)^3}{10s}
		\end{equation*}
		\begin{equation*}
			v_m = 12m/s
		\end{equation*}
		
		B) Para calcular a velocidade instantânea precisamos primeiro derivar a função dada pela questão, pois a derivada do espaço é a velocidade instantânea.
		
		\begin{equation*}
			\vec{v} = \dfrac{d\vec{x}(t)}{dt}
		\end{equation*}
		
		Derivando vamos obter a equação da seguinte forma:
		
		\begin{equation*}=
			\vec{v}(t) = 4,80 \cdot m/s^2 \cdot t - 0,36 \cdot m/s^3 \cdot t^2
		\end{equation*}
		
		i) Quando o $t=0s$ é 0 (zero), pois todos os termos vão se multiplicar por zero.
		
		ii) Verificando quando o $t=5,0s$
		
		\begin{equation*}
			\vec{v}(5,0s) = 4,80 \cdot m/s^2 \cdot 5,0s - 0,36 \cdot m/s^3 \cdot (5,0s)^2
		\end{equation*}
		\begin{equation*}
			\vec{v}(5,0s) = 15,00 m/s
		\end{equation*}
		
		iii) Verificando quando o $t=10,0s$
		
		\begin{equation*}=
			\vec{v}(10,0s) = 4,80 \cdot m/s^2 \cdot 10,0s - 0,36 \cdot m/s^3 \cdot (10,0s)^2
		\end{equation*}
		\begin{equation*}
			\vec{v}(10,0s) = 12,00 m/s
		\end{equation*}
		
		Quando o $t=0s$ é zero, quando $t=5,0s$ a velocidade instantânea é  15,00m/s e quando o $t=10,0s$ a velocidade instantânea é 12,00m/s.
		
		C) Para achar quando o carro fica em repouso, basta substituir a velocidade dele igual a zero e isolar o tempo, ou seja:
		
		\begin{equation*}
			0 = 4,80 \cdot m/s^2 \cdot t - 0,36 \cdot m/s^3 \cdot t^2
		\end{equation*}
		
		Resolvendo a equação obtermos
		
		\begin{equation*}
			t \approx 13,33s 
		\end{equation*}
		
		O carro retorna ao repouso novamente depois de aproximadamente 13,33 segundos depois da sua partida.
		
		\vspace{2em}
		
		\textbf{4.} a velocidade de um carro em função do tempo é dada por $v_x(t) = \alpha + \beta \cdot t^2$, onde $\alpha = 3,0m/s$ e $\beta = 0,100 m/s^3$. A) Calcule a aceleração média do carro para o intervalo de tempo de t = 0 a t = 5,0 s. b) Calcule a aceleração instantânea para i) t = 0 a t = 5,0 s. c) desenhe gráficos acurados $v \times t$ e $a \times t$ para o movimento do carro entre t = 0 e t = 5,0s.
		
		\textbf{resposta:}
		
		A) Para calcular a aceleração média do carro no intervalo de tempo t = 0 até t = 5,0s precisamos da equação seguinte:
		
		\begin{equation*}
			a_m = \frac{\Delta{\vec{v_x}}}{\Delta{t}} = \frac{v_x(t) - v_x(t_0)}{t - t_0}
		\end{equation*}
		
		Substituindo os valores para os valores do intervalos desejado a equação fica da seguinte forma
		
		\begin{equation*}
			a_m = \frac{v_x(5,0s) - v_x(0s)}{5,0s - 0s}   
		\end{equation*}
		\begin{equation*}
			a_m = \frac{\cancel{3,0m/s} + 0,100m/s^3 \cdot (5,0s)^{\cancel{2}} \cancel{- 3,0m/s}}{ \cancel{5,0s} }   
		\end{equation*}
		\begin{equation*}
			a_m = 0,100m/s^{\cancel{3}} \cdot 5,0 \cancel{s} = 0,5 m/s^2  
		\end{equation*}
		
		A aceleração média de t=0s até t=10s é de $0,5m/s^2$.
		
		B) Para calcular a velocidade instantânea precisamos derivar a função $v_x(t)$.
		
		\begin{equation*}
			\vec{a} = \dfrac{d\vec{v}}{dt} = 0,200m/s^3 \cdot t
		\end{equation*}
		
		i) Não precisa fazer conta quando o t=0s, pois irá multiplicar a equação por zero e o resultado é $0m/s^2$.
		
		ii) Quando o t=5,0s o resultado é $1m/s^2$.
		\begin{equation*}
			\vec{a} = 0,200m/s^{\cancel{3}} \cdot 5,0\cancel{s} = 1m/s^2
		\end{equation*}
		
		C)  
		
		i) Plotar o gráfico $v \cdot t$ da função $v(t)=0,100m/s^3 \cdot t+3m/s$.
		\begin{center}
			\begin{tikzpicture}
				\begin{axis}[
					domain=0:5, % Define o domínio da função
					xmin=0, xmax=5, ymin=0, ymax=5, % Limites dos eixos
					xlabel={\textit{t}}, ylabel={\textit{v(t)}}, % Rótulos dos eixos
					grid=both, % Adiciona uma grade
					]
					\addplot[blue, thick] {3+0.1*x^2}; % Plota a função
				\end{axis}
			\end{tikzpicture}
		\end{center}
		
		ii) Desenhar o gráfico $a \cdot t$ $a(t) = 0,200m/s^3 \cdot t$
		\begin{center}
			\begin{tikzpicture}
				\begin{axis}[
					domain=0:5, % Define o domínio da função
					xmin=0, xmax=5, ymin=0, ymax=2, % Limites dos eixos
					xlabel={\textit{t}}, ylabel={\textit{a(t)}}, % Rótulos dos eixos
					grid=both, % Adiciona uma grade
					]
					\addplot[red, thick] {0.2*x}; % Plota a função
				\end{axis}
			\end{tikzpicture}
		\end{center}
		
		\vspace{2em}
		
		\textbf{5.} O corpo humano pode sobreviver a um trauma por acidente com aceleração negativa (parada súbita) quando o módulo de aceleração é menor que 250 m/s². Suponha que você sofra um acidente de automóvel com velocidade inicial de 105 km/h e seja amortecido por um air bag que infla automaticamente. Qual deve ser a distância que o air bag se deforma para que você consiga sobreviver?
		
		\textbf{resposta:}
		
		Como não possuimos o tempo, para resolver vamos utilizar a equação de Torricelli.
		
		\begin{equation*}
			v^2 = v_0^2+2 \cdot a \cdot \Delta{s}
		\end{equation*}
		
		Para descorbri a distância, precisamos isolar ela na equação, ficando da seguinte forma:
		
		\begin{equation*}
			\Delta{s} = \frac{v^2 - v_0^2}{2 \cdot a}
		\end{equation*}
		
		A velocidade final será igual a zero, pois queremos quando estiver em repouso ou "batido".
		
		\begin{equation*}
			\Delta{s} = \frac{0^2 - (30m/s)^2}{-2 \cdot 250m/s^2}
		\end{equation*}
		\begin{equation*}
			\Delta{s} = \frac{900m^2/s^2}{500m/s^2}
		\end{equation*}
		\begin{equation*}
			\Delta{s} = \frac{900}{500} \cdot \frac{m^{\cancel{2}}}{\cancel{s^2}} \cdot \frac{\cancel{s^2}}{\cancel{m}}
		\end{equation*}
		\begin{equation*}
			\Delta{s} = 1,8 m
		\end{equation*}
		
		Para uma distância segura, o \textit{air bag} deve ser acionado ao uma distância de 1,8m antes da batida.
		
		\vspace{2em}
		
		\textbf{6.} Um trem de metrô parte do repouso em uma estação e acelera com uma taxa constante de 1,60 m/s² durante 14,0 s. Ele viaja com velocidade constante durante 70,0 s e reduz a velocidade com uma taxa constante de 3,50 m/s² até parar na estação seguinte. Calcule a distância total percorrida.
		
		\textbf{resposta:}
		
		i) Primeiro precisamos achar a distância percorrida durante a aceleração, que o trem partiu do repouso e teve uma aceleração de $1,60m/s^2$ por 14,0 segundos. O deslocamento pode ser tratado como um movimento retilíneo uniforme, e com isso podemos utilizar a equação horária do espaço, que descrevemos da seguinte forma:
		
		\begin{equation*}
			s_a(t) = s_0 + v_0 \cdot t + \frac{1}{2} \cdot a \cdot t^2
		\end{equation*}
		
		Onde $s_0 = 0m$, $v_0 = 0m/s$, $a = 1,6m/s^2$ e $s_a(t)$ é a posição do trem durante a acelração.
		
		\begin{equation*}
			s_a(t) = \cancel{s_0} + \cancel{v_0} \cdot t + \frac{1}{2} \cdot a \cdot t^2
		\end{equation*}
		\begin{equation*}
			s_a(t) =\frac{1}{2} \cdot a \cdot t^2
		\end{equation*}
		
		Agora podemos verificar a distância percorrida pelo trem até os 17 segundos.
		
		\begin{equation*}
			s_a(17,0s) =\frac{1}{2} \cdot a \cdot (17,0s)^2
		\end{equation*}
		\begin{equation*}
			s_a(17,0s) \approx 156,8m
		\end{equation*}
		
		ii) Precisamos achar a velocidade que ele atingio até os 17,0 segundos para saber a velocidade que ele manteve constante por mais 70 segundos. Para isso vamos utlizar a equação de Torricelli e isolar a velocidade final.
		
		\begin{equation*}
			v^2 = v_0^2 + 2 \cdot a \cdot \Delta{s}
		\end{equation*}
		
		Onde $v_0=0$, $a=1,60m/s^2$ e o $\Delta{s} = 156,8m$. Substituindo na equação e tirando a raiz fica da segunte forma:
		
		\begin{equation*}
			v^2 = \cancel{v_0}^2 + 2 \cdot 1,60m/s^2 \cdot 156,8m
		\end{equation*}
		\begin{equation*}
			v^2 = 501,76m^2/s^2 = \sqrt{501,76m^2/s^2}
		\end{equation*}
		\begin{equation*}
			v_c \approx 22,39m/s  
		\end{equation*}
		
		Descobrindo a velocidade que ele obtem quando chega em 17,0 segundos, essa velocidade é a velocidade constante que ele irá se manter por mais 70,0 segundos. Podemos utilizara equação horária do espaço, considerado novamente que ele irá começar na distância zero.
		
		\begin{equation*}
			s_c(t)=\cancel{s_0}+v_c \cdot t
		\end{equation*}
		
		Onde $v_c = 22,39m/s$ e precisamos ver sua posição no tempo igual a 70,0 segundos.
		
		\begin{equation*}
			s_c(70,0s)=22,39\frac{m}{\cancel{s}} \cdot 70,0\cancel{s} = 1.724,1m
		\end{equation*}
		
		iii) Agora precisamos entrar a distância que ele percorreu ao começar a desaceleração e entrar em repouso. Também será utilizado a equação de Torricelli para resolver. Onde $v_0=22,39m/s$, $v^2=0m/s$, $a=-3,50m/s^2$ e o $\Delta{s}_r$ é a constante que iremos isolar e tentar descobrir.
		
		\begin{equation*}
			0^2=(22,39m/s)^2 - 2 \cdot -3,50m/s^2 \cdot \Delta{s} _r
		\end{equation*}
		
		Isolando o $\Delta{s} _r$ na equação
		
		\begin{equation*}
			\Delta{s} _r = \frac{-(22,39m/s)^2}{-2 \cdot 3,50 m/s^2} = 72,90m
		\end{equation*}
		
		Para descobrir a distância total percorrida, precisamo somar a distância de aceleração mais a distância que ele se manteve constante mais a distância em que ele desacelerou. Descrevemos a equação da seguinte forma
		
		\begin{equation*}
			\Delta{s}_t = s_a + s_c + \Delta{s}
		\end{equation*}        
		
		Onde $s_a = 156,8m$, $s_c=1.724,1m$ e $\Delta{s} _r = 72,90m$.
		
		\begin{equation*}
			\Delta{s}_t = 156,8m + 1.724,1m + 72,90m = 1.797,00m
		\end{equation*}
		
		A distância total percorrida pelo trem foi de 1.797,00m.
		
		\vspace{2em}
		
		\textbf{8.} Um ovo é atirado verticalmente de baixo para cima de um ponto máximo da cornija na extremidade superior de um edifício alto. Ele passa rente a cornija em seu movimento para baixo, atingindo um ponto a 50,0m abaixo da cornija 5,0s após deixar a mão do lançador. Despreze a esistência do ar. a) Calcule a velocidade inicial do ovo. b) Qual a altura máxima atingida acima do ponto inicial do lançamento? c) Qual o módulo da velocidade nessa altura máxima? d) Qual é o módulo e o sentido da aceleração nessa altura máxima? Faça gráficos a x t, v x t e y x t para o movimento do ovo.
		
		\textbf{resposta:}
		
		\textbf{9.} A aceleração de uma motocicleta é dada por $a_x(t) = At - Bt^2$ onde $A = 1,5m/s^3$ e $B = 0,120 m/s^4$. A motocicleta está em repouso na origem no instante $t = 0$. a) Calcule sua velocidade e posição em função do tempo. b) Calcule a velocidade máxima que ela pode atingir.
		
		\textbf{resposta:}
		
		Para descobrir a velocidade e a posição precisamos intergrar a aceleração.
		
		Para descorbri a velocidade precisamos resolver a igualdade abaixo
		\begin{equation*}
			v_x(t) = \int_{t_0}^{t_1} a_x(t)dt
		\end{equation*}
		
		Igualdade resolvida:
		
		\begin{equation*}
			v_x(t) = \frac{A \cdot t^2}{2} - \frac{B \cdot t^3}{3}
		\end{equation*}
		
		Para descobrir a função do espaço precisamos intergrar a velocidade.
		
		\begin{equation*}
			s_x(t) = \int_{t_0}^{t_1} v_x(t)dt
		\end{equation*}
		
		Igualdade resolvida:
		
		\begin{equation*}
			s_x(t) = \frac{A \cdot t^3}{6} - \frac{B \cdot t^4}{12}
		\end{equation*}
		
		A) Para calcular a velocidade e a posição em relação ao tempo precisamos desenhar o gráfico para um melhor entendimento de como foi o movimento dele.
		
		i)O gráfico da equação $v_x(t) = \frac{A \cdot t^2}{2} - \frac{B \cdot t^3}{3}$ da velocidade em função do tempo.
		
		\begin{center}
			\begin{tikzpicture}
				\begin{axis}[
					domain=0:20, % Define o domínio da função
					xmin=0, xmax=20, ymin=0, ymax=40, % Limites dos eixos
					xlabel={\textit{t}}, ylabel={\textit{v(t)}}, % Rótulos dos eixos
					grid=both, % Adiciona uma grade
					]
					\addplot[red, thick] {((4.5*x^2)-(0.24*x^3))/6}; % Plota a função
				\end{axis}
			\end{tikzpicture}
		\end{center}
		
		ii) O gráfico da equação $s_x(t) = \frac{A \cdot t^3}{6} - \frac{B \cdot t^4}{12}$ do espaço em função do tempo da equação .
		\begin{center}
			\begin{tikzpicture}
				\begin{axis}[
					domain=0:20, % Define o domínio da função
					xmin=0, xmax=15, ymin=0, ymax=30, % Limites dos eixos
					xlabel={\textit{t}}, ylabel={\textit{s(t)}}, % Rótulos dos eixos
					grid=both, % Adiciona uma grade
					]
					\addplot[blue, thick] {((3*x^3)-(0.24*x^4))/24}; % Plota a função
				\end{axis}
			\end{tikzpicture}
		\end{center}
		
		\vspace{2em}
		
		\textbf{10.} Você está no telhado de um prédio da UFMT 46 m acima do solo. Seu professor, que possui 1,80 m de altura, está caminhando próximo do edifício com um velocidade constante de 1,20 m/s. Se você desejar jogar um ovo na cabeça dele, em que ponto ele deve estar quando você largar o ovo? Suponha que o ovo esteja em queda livre e que o professor caminhe em linha reta em direção a porta do edifício
		
		\textbf{resposta:}
		
		A altura total é da ponta da da cabeça do professor até a ponta do edifício. O $\Delta{s}$ total é 46 metros, como a altura do professor é 1,8m, a altura de queda será a altura total menos a altura do professor.
		
		\begin{equation*}
			h_q = \Delta{s} - 1,8 = 44,2m
		\end{equation*}
		
		Primeiro precisamos saber o tempo de queda livre do ovo. Para isso utilizamos a seguinte fórmula:
		
		\begin{equation*}
			x_q(t) = x_0 + v_0 \cdot t + \frac{a_y \cdot t^2}{2}
		\end{equation*}
		
		Onde $x_q(t) = h_q = 44,2m$, $x_0 = 0$, $v_0 = 0$, $a_y = g = 9,81 m/s^2$ e $t = t_q$ que significa tempo de queda. Substituindo esses valores na fórmula e resolvendo conseguimos o seguinte resultado
		
		\begin{equation*}
			44,2m = \cancel{x_0} + \cancel{v_0 \cdot t_q} + \frac{9,81 \cdot t_q^2}{2}
		\end{equation*}
		
		\begin{equation*}
			t_q \approx 3,0s
		\end{equation*}
		
		Agora precisamos descobrir a distância que o professor percorre durante 3,0 segundos. Como se trata de um movimento retilíneo e uniforme, podemos utilizar a seguinte fórmula
		
		\begin{equation*}
			x(t) = x_0 + v \cdot t
		\end{equation*}
		
		Onde $x(t) = x_p$, $x_0 = 0$, $v = 1,2m/s$ e $t = t_q = 3,0s$. substituindo os valores na fórmula podemos resolver e descobrir a distância percorrida pelo professor durante o tempo de queda
		
		\begin{equation*}
			x_p = \cancel{x_0} + 1,2m/\cancel{s} \cdot 3,0\cancel{s} 
		\end{equation*}
		\begin{equation*}
			x_p = 3,60m 
		\end{equation*}
		
		Então o ovo deve ser solto quando o professor estiver 3,6 metros de distância.
		
		\vspace{2em}
		
		\textbf{11.} Se $\vec{r} = bt^2 \hat{i} + c t^3 \hat{j}$, onde b e c são constantes positivas, quando o vetor velocidade faz um ângulo de $45,0^\circ$ com os eixos Ox e Oy?
		
		\textbf{resposta:}
		
		Para achar o vetor de velocidade precisamos derivar o vetor de posição em relação ao tempo.
		
		\begin{equation*}
			\vec{v} = \dfrac{d\vec{r}}{dt} = \dfrac{bt^2\hat{i} + ct^3\hat{j}}{dt}
		\end{equation*}
		\begin{equation*}
			\vec{v} = 2bt\hat{i} + 3ct^2\hat{j}
		\end{equation*}
		
		Para ter um ângulo de 45 graus, o triângulo retângulo precisa ser necessáriamente isósceles e a hipotenusa diferente. Então sabemos que os vetores de velocidade de $v_x\hat{i} = v_y\hat{j}$.
		
		\begin{equation*}
			2b\cancel{t} = 3ct^{\cancel{2}}
		\end{equation*}
		\begin{equation*}
			t = \frac{2b}{3c}
		\end{equation*}
		
		\vspace{2em}
		
		\textbf{12.} Um avião a jato está voando a uma altura constante. No instante $t_1 =0$, os componentes da velocidade são $v_x = 90m/s$, $v_y = 110 m/s$. No instante $t_2 = 30,0s$ os componentes são $v_x = -170 m/s$, $v_y = 40 m/s$. a) Faça um esboço do vetor velocidade para $t_1$ e para $t_2$. Qual a diferença entre esses dois vetores? Para este intervalo de tempo, calcule b) os componentes da aceleração média, c) o módulo, a direção e o sentido da aceleração média.
		
		\textbf{resposta:}
		
		A) Esboço dos vetores de $t_1$ e $t_2$
		
		\begin{center}
			\begin{tikzpicture}
				% Eixos
				\draw[->] (0,0) -- (5,0) node[right] {$v_x$};
				\draw[->] (0,0) -- (0,5) node[above] {$v_y$};
				
				% Vetor v1
				\draw[->, blue, thick] (0,0) -- (3,4.4) node[right] {$\mathbf{v_1}$};
				
				% Vetor v1_y
				\draw[->, blue, thick] (0,0) -- (0,4.4) node[left] {$\mathbf{v_{x1}}$};
				
				\draw[-, black, dashed] (3,4.4) -- (0,4.4);
				
				% Vetor v1_x
				\draw[->, blue, thick] (0,0) -- (3,0) node[above] {$\mathbf{v_{x1}}$};
				
				\draw[-, black, dashed] (3,0) -- (3,4.4);
				
				% Vetor v2
				\draw[->, red, thick] (0,0) -- (-3.8,1.5) node[below left] {$\mathbf{v_2}$};
				
				% Vetor v2_y
				\draw[->, red, thick] (0,0) -- (-3.8,0) node[left] {$\mathbf{v_{x2}}$};
				
				\draw[-, black, dashed] (-3.8,0) -- (-3.8,1.5);
				
				% Vetor v2_x
				\draw[->, red, thick] (0,0) -- (0,1.5) node[right] {$\mathbf{v_{x2}}$};
				
				\draw[-, black, dashed] (-3.8,1.5) -- (0,1.5);
				
				% Marcação dos vetores
				\draw[dashed] (0,0) -- (3,0);
				\draw[dashed] (0,0) -- (0,4.4);
				\draw[dashed] (0,0) -- (-3.8,0);
				\draw[dashed] (0,0) -- (0,1.5);
			\end{tikzpicture}
		\end{center}
		Para encontrar a difereça precisa fazer a velocidade inal menos o velocidade inicial da seguinte forma
		
		\begin{equation*}
			\Delta{v_x} = v_{2x} - v_{1x} = -170m/s - 90m/s = -260m/s
		\end{equation*}
		\begin{equation*}
			\Delta{v_y} = v_{2y} - v_{1y} = 40m/s - 110m/s = -70m/s
		\end{equation*}
		
		Então
		
		\begin{equation*}
			\Delta{v} = (-260m/s , -70m/s)
		\end{equation*}
		
		B) Para achar a velocidade média precisamos utilizar a seguinte fórmula:
		
		\begin{equation*}
			a_m = \frac{\vec{v_2} - \vec{v_1}}{t_2 - t_1}
		\end{equation*}
		
		Aceleração média no eixo x:
		\begin{equation*}
			a_{mx} = \frac{-170 m/s - 90 m/s}{30,0 s - 0s} = \frac{-260m/s}{30s}
		\end{equation*}
		\begin{equation*}
			a_{mx} \approx - 8,67 m/s^2
		\end{equation*}
		
		Aceleração média no eixo y:
		\begin{equation*}
			a_{my} = \frac{40m/s - 110m/s}{30,0 s - 0s} = \frac{-70m/s}{30s}
		\end{equation*}
		\begin{equation*}
			a_{my} \approx - 2,33 m/s^2
		\end{equation*}
		
		C) Para calcular o módulo de um vetor precisamos utilizar a fórmula de pitágoras
		
		\begin{equation*}
			|a_m| = \sqrt{(a_{mx})^2 + (a_{my})^2} = \sqrt{(-8,67)^2 + (-2,33)^2} \approx 8,92 m/s
		\end{equation*}
		
		Agora precisamos achar o ângulo. Para isso é necessário usar regras dos ângulos.
		
		\begin{equation*}
			\angle^{\circ} = \arctan(\frac{a_{my}}{a_{mx}}) = \arctan (\frac{-2,33}{-8,67}) = \arctan (0,27) \approx 15,11^{\circ}  
		\end{equation*}
		
		O módulo da aceleração média é aproximadamente $8,92 m/s^2$ e a angulação é aproximadamente $15,11^{\circ}$.
		
		\vspace{2em}
		
		\textbf{13.} Dois grilos, Chirpy e Milada, saltam do topo de um rochedo íngreme. Chirpy simplesmente se deixa cair e chega ao solo em 3,50 s, enquanto Milada salta horizontalmente com velocidade inicial de 95,0 cm/s. A que distância da base do rochedo Milada vai atingir o chão?
		
		\textbf{resposta:}
		
		\vspace{2em}
		
		\textbf{14.} Uma bola de gude rola horizontalmente com velocidade escalar $v_0$ e cai do topo de uma plataforma de 2,75m de altura, sem sofrer nenhuma resistência significativa do ar. No nível do solo, a 2,0m da base da plataforma, há um buraco escancarado de diâmetro de 1,50m. Para qual alcance da velocidade de $v_0$ a bola de gude aterrissará no buraco?
		
		\textbf{resposta:}
		
		Primeiro precisamos calcular  o tempo de queda da bola. Para isso será necessário utilizar a seguinte fórmula
		
		\begin{equation*}
			y(t) = y_0 + v_{y0} \cdot t + \frac{a_y \cdot t^2}{2}
		\end{equation*}
		
		Onde $y(t) = 0$, $y_0 = 2,75m$, $v_{y0} = 0$, $a_y = -g = 9,81 m/s^2$. Substituindo esses valores na equação vamos conseguir calcular o tempo de queda da bolinha
		
		\begin{equation*}
			0 = 2,75m + \cancel{v_{y0} \cdot t} - \frac{g \cdot t^2}{2}
		\end{equation*}
		\begin{equation*}
			t = \sqrt{\frac{2 \cdot 2,75 m}{9,81 m/s^2}} \approx 0,56s
		\end{equation*}
		
		Achando o tempo de queda, podemos utilizar a seguinte equação e isolar a velocidade e substituir o t por 0,56.
		
		\begin{equation*}
			x(t) = x_0 + v_{x0} \cdot t
		\end{equation*}
		
		Onde $x(t) = 3,0m$ pois a bolinha pode cair uma distância menor que 3,5m e maior que 2,0m, $x_0=0$ e $t=3,0s$ 
		
		\begin{equation*}
			3,0m = \cancel{x_0} + v_{x0} \cdot 0,56s
		\end{equation*}
		\begin{equation*}
			v_{x0} = \frac{3,0m}{0,56s} \approx 5,35 m/s
		\end{equation*}
		
		\vspace{2em}
		
		\textbf{15.} Um avião voa a uma velocidade de 90,0 m/s a um ângulo de 23,0º acima da horizontal. Quando está a 114 m diretamente sobre um cachorro parado no nível do solo, uma mala cai do compartimento de bagagens. A que distância do cachorro a mala vai cair? Despreze a resistência do ar.
		
		\textbf{resposta:}
		
		Primeiro precisamos calcular  o tempo de queda da mala. Para isso será necessário utilizar a seguinte fórmula
		
		\begin{equation*}
			y(t) = y_0 + v_{y0} \cdot t + \frac{a_y \cdot t^2}{2}
		\end{equation*}
		
		Onde $y(t) = 0$, $y_0 = 114m$, $v_{y0} = 0$, $a_y = -g = 9,81 m/s^2$. Substituindo esses valores na equação vamos conseguir calcular o tempo de queda da bolinha
		
		\begin{equation*}
			0 = 114m + \cancel{v_{y0} \cdot t} - \frac{9,81m/s^2 \cdot t^2}{2}
		\end{equation*}
		\begin{equation*}
			t = \sqrt{\frac{2 \cdot 144m}{9,81 m/s^2}} \approx 5,41s
		\end{equation*}
		
		Achando o tempo de queda, podemos utilizar a seguinte equação e isolar a velocidade e substituir o t por 5,41s.
		
		\begin{equation*}
			x(t) = x_0 + v_{x0} \cdot t
		\end{equation*}
		
		Onde $x(t) = x$, $x_0=0$ $90,0m/s$ e $t=5,41s$ 
		
		\begin{equation*}
			x = \cancel{x_0} + 90,0m/\cancel{s} \cdot 5,41\cancel{s}
		\end{equation*}
		\begin{equation*}
			x \approx 486,9 m
		\end{equation*}
		
		A distância que a caixa irá cair do cachorro é de 486,9 metros.
		
		\vspace{2em}
		
		\textbf{16.} Em um teste de um ‘aparelho para g’, um voluntário gira em um círculo horizontal de raio igual a 7,0 m. Qual é operíodo da rotação para que a aceleração centrípeta possua módulo de a) 3,0g? b) 10g?
		
		\textbf{resposta:}
		
		Para resolver esse problema, precisamos usar a fórmula da aceleração centrípeta e a relação entre a aceleração centrípeta e o período de rotação.
		
		\begin{equation*}
			a_c = \frac{v^2}{R}
		\end{equation*}
		
		Onde o $v$ é a velocidade tengencial e $R$ é o raio.
		
		A velocidade tangencial $v$ pode ser relacionada ao período $T$ pela fórmula:
		
		\begin{equation*}
			v = \frac{2 \cdot \pi \cdot R}{T}
		\end{equation*}
		
		Substituindo essa expressão na fórmula da aceleração centrípeta, obtemos:
		
		\begin{equation*}
			a_c = \frac{4 \cdot \pi^2 \cdot R}{T^2}
		\end{equation*}
		
		Onde $a_c = $ aceleração centrípeta, $R = 7,0m$, $T = periodo$. Rearranjando a fórmula para encontrar o período T, temos:
		
		\begin{equation*}
			T = \sqrt{\frac{4 \cdot \pi^2 \cdot R}{a_c}}
		\end{equation*}
		
		i) Substituindo o $a_c = 3,0 \cdot 9,8 m/s^2$.
		
		\begin{equation*}
			T = \sqrt{\frac{4 \cdot 9,87 \cdot 7,0m}{29,4m/s^2}}
		\end{equation*}
		\begin{equation*}
			T \approx 3,06s
		\end{equation*}
		
		i) Substituindo o $a_c = 10,0 \cdot 9,8 m/s^2$.
		
		\begin{equation*}
			T = \sqrt{\frac{4 \cdot 9,87 \cdot 7,0m}{98m/s^2}}
		\end{equation*}
		\begin{equation*}
			T \approx 1,68s
		\end{equation*}
		
		Então, os períodos para as acelerações centrípetas especificadas são aproximadamente 3,06s para 3,0g e 1,68s para 10g.
		
		\vspace{2em}
		
		\textbf{17.}  Uma roda-gigante possui raio de 14,0 m e gira no sentido anti-horário. Em dado instante, um passageiro na periferia da roda e passando no ponto mais baixo do movimento circular, move-se a 3,0 m/s e está ganhando velocidade com uma taxa de 0,500 m/s². a) Determine o módulo, a direção e o sentido da aceleração do passageiro nesse instante. b) faça um desenho da roda-gigante e do passageiro, mostrando a velocidade e os vetores de aceleração dele.
		
		\textbf{resposta:}
		
		\vspace{2em}
		
		\textbf{18.}  Uma canoa possui velocidade de 0,40 m/s do sul para o leste em relação a Terra. A canoa se desloca em um rio que escoa a 0,50 m/s do oeste para leste em relação a Terra. Determine o módulo, a direção e o sentido da velocidade da canoa em relação ao rio.
		
		\textbf{resposta:}
		
		Para determinar a velocidade da canoa em relação ao rio, precisamos fazer uma análise vetorial das velocidades envolvidas. Vamos decompor e calcular as velocidades relativas.
		
		\begin{equation*}
			\vec{v}_{canoa/Terra} = 0,40m/s
		\end{equation*}
		\begin{equation*}
			\vec{v}_{rio/Terra} = 0,50m/s
		\end{equation*}
		
		i) Decomposição dos vetores
		
		1) Velocidade da canoa em relação a terra:
		\begin{equation*}
			\vec{v}_{canoa/Terra} = (0,40m/s, 0)
		\end{equation*}
		
		2) Velocidade do rio em relação a terra:
		\begin{equation*}
			\vec{v}_{canoa/Terra} = (0,50m/s, 0)
		\end{equation*}
		
		ii) Determinando a velocidade da canoa em relação ao rio
		
		2) Velocidade do rio em relação a terra:
		\begin{equation*}
			\vec{v}_{canoa/rio} = \vec{v}_{canoa/Terra} - \vec{v}_{canoa/Terra}
		\end{equation*}
		\begin{equation*}
			\vec{v}_{canoa/rio} = (0,40m/s, 0) - (0,50m/s, 0) = (-0,10m/s,0)
		\end{equation*}
		
		iii) O módulo da canoa em realção ao rio
		
		\begin{equation*}
			|\vec{v}_{canoa/rio}| = \sqrt{(-0,10)^2 + 0^2} = 0,10m/s
		\end{equation*}
		
		iv) Direção e sentido do vetor da canoa em relação ao rio. O vetor $\vec{v}_{canoa/rio}$ é $-0,10m/s$ ao longo do eixo x, o que indica que a canoa está se movendo 0,10 m/s para o oeste em relação ao rio.
		
		\vspace{2em}
		
		\textbf{19.} O piloto de um avião deseja voar de leste para oeste. Um vento de 80,0 km/h sopra do norte para o sul. a) Se a velocidade do avião em relação ao ar é igual a 320 km/h qual deve ser a direção escolhida pelo piloto? b) Qual é a velocidade do avião em relação ao solo? Ilustre sua solução com um diagrama vetorial.
		
		\textbf{resposta:}		
			
			Para voar do leste para o oeste, precisa das componente resultante da velocidade em relação ao solo deve estar apenas em direção ao oeste.
			
			A) Direção que o piloto deve escolher para que o avião voe de leste para oeste, a componente norte-sul da velocidade do avião em relação ao ar deve anular a velocidade do vento do norte para o sul.
			
			Os vetores $\vec{v}_\text{vento}$ é um vetor de intesidade 80 km/h apontando para o sul (eixo negativo $y$), o $\vec{v}_\text{avião/ar}$ é um vetor de intesidade 320 km/h com um ângulo 
			$\theta$ ao norte do oeste.
			
			A velocidade do avião em relação ao solo deve ser puramente oeste, ou seja, a componente norte-sul $y$ da velocidade resultante deve ser zero.
			
			Portanto, a componente $y$ da velocidade do avião em relação ao ar deve cancelar a componente do vento:
			
			\begin{equation*}
				\sin{\theta} = \frac{80}{320} = \arcsin({\frac{1}{4}})			
			\end{equation*}
			\begin{equation*}
				\sin{\theta} \approx 14,48^\circ
			\end{equation*}
			
			O ângulo $\theta$ encontrado é o ângulo de norte do oeste.
			
			B) Agora, calculamos a velocidade resultante do avião em relação ao solo (componente $x$, leste-oeste):

			\begin{equation*}
				\cos(\theta) = \cos(14,48^\circ)	
			\end{equation*}
						
			Usando a identidade trigonométrica e sabendo que $\cos(14,48^\circ) \approx 0,970$
			
			\begin{equation*}
				\vec{v}_\text{avião/solo} = 320 \cdot \cos(14,48^\circ)	
			\end{equation*}
			\begin{equation*}
				\vec{v}_\text{avião/solo} \approx 320 \cdot 0,970 \approx 310,4 \, \text{km/h}	
			\end{equation*}
			
			
			Diagrama dos vetores (Coloquei porque eu queria aprender vetor no LaTeX)
			\begin{center}
				\begin{tikzpicture}
					% Vetor de velocidade do avião/ar
					\draw[->, thick] (0,0) -- (3,1) node[midway, above] {$\vec{v}_\text{avião/ar}$};
					% Vetor de velocidade do vento
					\draw[->, thick] (0,0) -- (0,-1) node[midway, left] {$\vec{v}_\text{vento}$};
					% Vetor resultante do avião/solo
					\draw[->, thick] (0,0) -- (3,0) node[midway, below] {$\vec{v}_\text{avião/solo}$};
					% Marca o ângulo theta
					\draw (0.8,0) arc[start angle=0,end angle=18.43,radius=0.8];
					\node at (0.9,0.3) {$\theta$};
				\end{tikzpicture}	
			\end{center}
		
		\vspace{2em}
		
		\textbf{20.}  Um professor de física faz loucas proezas em suas horas vagas. Sua última façanha foi saltar sobre um rio com a sua motocicleta. A rampa de decolagem era inclinada de 53,0º, a largura do rio era de 40,0 m, e a outra margem estava a 15,0m abaixo do nível da rampa. O rio estava a 100 m abaixo do nível da rampa. Despreze a resistência do ar. a) Qual deveria ser sua velocidade para que ele pudesse alcançar a outra margem sem cair no rio? b) Caso sua velocidade fosse igual à metade do valor encontrado em (a), aonde ele cairia?
		
		\textbf{resposta:}
	
		Equações de movimento para o salto:
		\begin{equation*}
			y(t) = v_0 \sin(\theta) t - \frac{1}{2}gt^2
		\end{equation*}
		\begin{equation*}
			x(t) = v_0 \cos(\theta) t
		\end{equation*}
		
		Alcançar a outra margem significa que $y(t) = -h$ quando $x(t) = D$.
		
		Substituindo $t = \frac{D}{v_0 \cos(\theta)}$ na equação para $y(t)$ para achar a equação da trajetória
		
		\begin{equation*}
			-h = v_0 \sin(\theta) \frac{D}{v_0 \cos(\theta)} - \frac{g}{2} \left(\frac{D}{v_0 \cos(\theta)}\right)^2
		\end{equation*}
		
		Como $\frac{\sin(\theta)}{\cos(\theta)} = \tan(\theta)$, vamos simplificar essa parte na equação
		
		\begin{equation*}
			-h = D \tan(\theta) - \frac{gD^2}{2v_0^2 \cos^2(\theta)}
		\end{equation*}
		
		Além disso, vamos isolar o $v_0$ na equação da trajetória
		
		\begin{equation*}
			v_0^2 = \frac{gD^2}{2\cos^2(\theta)(D \tan(\theta) + h)}
		\end{equation*}
		
		Substituindo os valores que a questão nos fornece e resolvendo a igualdade, a equação deve ficar da seguinte forma
		
		\begin{equation*}
			v_0^2 = \frac{9,8 \cdot 40,0^2}{2 \cdot \cos^2(53,0^\circ) \cdot (40,0 \tan(53,0^\circ) + 15,0)}
		\end{equation*}
		
		\begin{equation*}
			v_0 \approx 21,8 \, \text{m/s}
		\end{equation*}	
		
		B) Queda com Metade da Velocidade, ou seja, com a velocidade $v_0 = 10,9 \text{m/s}$
		
		\begin{equation*}
			x(t) = 10,9 \cos(53,0^\circ) t
		\end{equation*}
		
		\begin{equation*}
			y(t) = 10,9 \sin(53,0^\circ) t - \frac{1}{2} \cdot 9,8 \cdot t^2
		\end{equation*}
		
		Resolvendo para quando $y(t) = -100$ que é a altura do rio
		
		\begin{equation*}
			-100 = 10,9 \sin(53,0^\circ) t - 4,9t^2
		\end{equation*}
		
		\begin{equation*}
			-100 = 8,7t - 4,9t^2
		\end{equation*}
		
		Somando +100 na desigualdade, nota-se que forma uma função quadrática. Para resolver a função quadrática se utiliza bhaskara. Como esta em função de t a equação fica
		
		\begin{equation*}
			t = \frac{-8,7 \pm \sqrt{8,7^2 - 4 \cdot (-100) \cdot (-4,9)}}{2 \cdot -4,9}
		\end{equation*}
		\begin{equation*}
			t \approx 4,7s	
		\end{equation*}
		
		Com o valor do tempo, podemos substituir na equação $x(t)$
		
		\begin{equation*}
			x(t) = 10,9 \cos(53,0^\circ) \cdot4,7
		\end{equation*}
		
		\begin{equation*}
			x(t) \approx 31,0m	
		\end{equation*}
		
	\end{flushleft}
\end{document}
